\documentclass[10pt]{article}
\usepackage{geometry}
\geometry{letterpaper,total={340pt,9.0in}}
%% Custom Page Layout Adjustments (use publisher page-geometry)
\usepackage{amsmath,amssymb}
\usepackage{tikz}
\usetikzlibrary{calc}
\usetikzlibrary{arrows}
\usetikzlibrary{decorations.pathreplacing}
\usetikzlibrary{patterns}
\usetikzlibrary{calc,intersections,through,backgrounds}
\usetikzlibrary{shapes.geometric}
\ifdefined\tikzset
\tikzset{ampersand replacement = \amp}
\fi
\newcommand\degree[0]{^{\circ}}
\newcommand\Ccancel[2][black]{\renewcommand\CancelColor{\color{#1}}\cancel{#2}}
\newcommand{\alert}[1]{\boldsymbol{\color{magenta}{#1}}}
\newcommand{\blert}[1]{\boldsymbol{\color{blue}{#1}}}
\newcommand{\bluetext}[1]{\color{blue}{#1}} 
\delimitershortfall-1sp
\newcommand\abs[1]{\left|#1\right|}
\newcommand{\lt}{<}
\newcommand{\gt}{>}
\newcommand{\amp}{&}
\begin{document}
\pagestyle{empty}
\resizebox{\width}{\height}{
\tikzset{%
}
\begin{tikzpicture} [scale=.3]
\draw[cyan, thin] (0,0) grid (20,20);
\draw[black,very thick] (0,0)--(20,0);
\draw[black,very thick] (0,0)--(0,20);
\foreach \x [evaluate=\x as \xi using int(\x+60)] in  {0,5,10,15,20} {
 \draw[cyan, very thick] (\x,0) --++(0,20);
 \draw[black] (\x,.15) --++(0,-.3)  node[below, yshift=-2, fill=white, inner sep=1]   {$\xi$};
}
\foreach \x [evaluate=\x as \xi using int(\x+55)] in  {0,5,10,15,20} {
 \draw[cyan, very thick] (0,\x) --++(20,0);
 \draw[black] (.15,\x) --++(-.3,0)  node[left, xshift=-2, fill=white, inner sep=1]   {$\xi$};
}
\filldraw[red] (6,4) circle (.3);
\filldraw[red] (6,16) circle (.3);
\filldraw[red] (8,8) circle (.3);
\filldraw[red] (8.5,7.5) circle (.3);
\filldraw[red] (8.5,10) circle (.3);
\filldraw[red] (10,9) circle (.3);
\filldraw[red] (10,10) circle (.3);
\filldraw[red] (10.5,11.75) circle (.3);
\filldraw[red] (11,7.5) circle (.3);
\filldraw[red] (11,10) circle (.3);
\filldraw[red] (12,11) circle (.3);
\filldraw[red] (12,12.75) circle (.3);
\filldraw[red] (13,12.75) circle (.3);
\filldraw[red] (14,7) circle (.3);
\filldraw[red] (14.5,13) circle (.3);
\filldraw[red] (15.5,14.5) circle (.3);
\node[font=\bf] at (10,-3) {Male Height (inches)};
\node[rotate=90, font=\bf] at (-3,10) {Female Height (inches)};
\end{tikzpicture}
}
\end{document}
