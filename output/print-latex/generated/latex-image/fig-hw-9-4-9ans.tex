\documentclass[10pt]{article}
\usepackage{geometry}
\geometry{letterpaper,total={340pt,9.0in}}
%% Custom Page Layout Adjustments (use publisher page-geometry)
\usepackage{amsmath,amssymb}
\usepackage{tikz}
\usetikzlibrary{calc}
\usetikzlibrary{arrows}
\usetikzlibrary{decorations.pathreplacing}
\usetikzlibrary{patterns}
\usetikzlibrary{calc,intersections,through,backgrounds}
\usetikzlibrary{shapes.geometric}
\ifdefined\tikzset
\tikzset{ampersand replacement = \amp}
\fi
\newcommand\degree[0]{^{\circ}}
\newcommand\Ccancel[2][black]{\renewcommand\CancelColor{\color{#1}}\cancel{#2}}
\newcommand{\alert}[1]{\boldsymbol{\color{magenta}{#1}}}
\newcommand{\blert}[1]{\boldsymbol{\color{blue}{#1}}}
\newcommand{\bluetext}[1]{\color{blue}{#1}} 
\delimitershortfall-1sp
\newcommand\abs[1]{\left|#1\right|}
\newcommand{\lt}{<}
\newcommand{\gt}{>}
\newcommand{\amp}{&}
\begin{document}
\pagestyle{empty}
\resizebox{\width}{\height}{
\tikzset{%
}
\begin{tikzpicture} [scale=.3]
\draw[cyan] (-10,-10) grid (10,10);
\draw[black,thick, ->, >=stealth'] (-10,0)--(10.6,0) node[right]{$x$};
\draw[black,thick, ->, >=stealth'] (0,-10)--(0,10.6) node[above]{$y$};
\foreach \x in {-10,-5,5,10} {
 \draw[black] (\x,.1) --++(0,-.2)  node[below, yshift=-2, fill=white, inner sep=1]   {$\x$};
 }
\foreach \x in {-10,-5,5,10} {
 \draw[black] (.1,\x) --++(-.2,0)  node[left, xshift=-2, fill=white, inner sep=1]   {$\x$};
 }
% \def\a{sqrt(32)};
\def\b{4};
\def\h{0};
\def\k{0};
\draw[black] ({\h-sqrt(32)},{\k-\b}) rectangle ({\h+sqrt(32)},{\k+\b});
\draw[samples=2,domain={-10:10},smooth,variable=\x,gray,dashed] plot ( {\x}, { \k+\b/sqrt(32) * (\x - \h) } );
\draw[samples=2,domain={-10:10},smooth,variable=\x,gray,dashed] plot ( {\x}, { \k-\b/sqrt(32) * (\x - \h) } );
\filldraw[black] (\h,\k) circle (.1);
\draw[samples=65,domain={-1.2:1.2},smooth,variable=\x,red,very thick, <->,>=stealth'] plot ( {\h+sqrt(32)*cosh(\x)}, {\k+\b*sinh(\x)});
\draw[samples=65,domain={-1.2:1.2},smooth,variable=\x,red,very thick, <->,>=stealth'] plot ( {\h-sqrt(32)*cosh(\x)}, {\k+\b*sinh(\x)});
\end{tikzpicture}
}
\end{document}
