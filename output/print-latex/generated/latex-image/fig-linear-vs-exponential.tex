\documentclass[10pt]{article}
\usepackage{geometry}
\geometry{letterpaper,total={340pt,9.0in}}
%% Custom Page Layout Adjustments (use publisher page-geometry)
\usepackage{amsmath,amssymb}
\usepackage{tikz}
\usetikzlibrary{calc}
\usetikzlibrary{arrows}
\usetikzlibrary{decorations.pathreplacing}
\usetikzlibrary{patterns}
\usetikzlibrary{calc,intersections,through,backgrounds}
\usetikzlibrary{shapes.geometric}
\ifdefined\tikzset
\tikzset{ampersand replacement = \amp}
\fi
\newcommand\degree[0]{^{\circ}}
\newcommand\Ccancel[2][black]{\renewcommand\CancelColor{\color{#1}}\cancel{#2}}
\newcommand{\alert}[1]{\boldsymbol{\color{magenta}{#1}}}
\newcommand{\blert}[1]{\boldsymbol{\color{blue}{#1}}}
\newcommand{\bluetext}[1]{\color{blue}{#1}} 
\delimitershortfall-1sp
\newcommand\abs[1]{\left|#1\right|}
\newcommand{\lt}{<}
\newcommand{\gt}{>}
\newcommand{\amp}{&}
\begin{document}
\pagestyle{empty}
\resizebox{\width}{\height}{
\tikzset{%
}
\begin{tikzpicture} [xscale=.4, yscale=.4] 
\draw[cyan] (0,0) grid (7,10);
\draw[black,thick,->,>=stealth'] (0,0)--(7.6,0) node[above]{$t$};
\draw[black,thick,->,>=stealth'] (0,0)--(0,10.6) node[left, xshift=1mm, yshift=1mm]{$y$};
\foreach \x  in {5} {
    \draw[black,thick] (\x,.15)--++(0,-.3) node[below, fill=white, inner sep=1, yshift=-2]{$\x$};
}
\foreach \x [evaluate=\x as \xi using int(10*\x)] in {5,10} {
    \draw[black,thick] (.15,\x)--++(-.3,0) node[left, fill=white, inner sep=1, xshift=-2]{$\xi$};
}
\draw[samples=65,domain={0:ln(20)/ln(2)},smooth,variable=\x,red,very thick, ->,>=stealth'] plot (\x, {1/2*2^(\x)});
\node[text=red,fill=white, inner sep=1] at (6.7,6.5) {$E(t)=5\cdot 2^t$};
\filldraw[red] (1,1) circle (.1);
\filldraw[red] (2,2) circle (.1);
\filldraw[red] (3,4) circle (.1);
\filldraw[red] (4,8) circle (.1);
\draw[blue,very thick, ->,>=stealth'] (0,1/2)--(7,19/10) node[above, xshift=-.2cm, yshift=2, fill=white, inner sep=1]{$L(t)=5+2t$};
\filldraw[blue] (1,7/10) circle (.1);
\filldraw[blue] (2,9/10) circle (.1);
\filldraw[blue] (3,11/10) circle (.1);
\filldraw[blue] (4,13/10) circle (.1);
\end{tikzpicture}
}
\end{document}
