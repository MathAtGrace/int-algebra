\documentclass[10pt]{article}
\usepackage{geometry}
\geometry{letterpaper,total={340pt,9.0in}}
%% Custom Page Layout Adjustments (use publisher page-geometry)
\usepackage{amsmath,amssymb}
\usepackage{tikz}
\usetikzlibrary{calc}
\usetikzlibrary{arrows}
\usetikzlibrary{decorations.pathreplacing}
\usetikzlibrary{patterns}
\usetikzlibrary{calc,intersections,through,backgrounds}
\usetikzlibrary{shapes.geometric}
\ifdefined\tikzset
\tikzset{ampersand replacement = \amp}
\fi
\newcommand\degree[0]{^{\circ}}
\newcommand\Ccancel[2][black]{\renewcommand\CancelColor{\color{#1}}\cancel{#2}}
\newcommand{\alert}[1]{\boldsymbol{\color{magenta}{#1}}}
\newcommand{\blert}[1]{\boldsymbol{\color{blue}{#1}}}
\newcommand{\bluetext}[1]{\color{blue}{#1}} 
\delimitershortfall-1sp
\newcommand\abs[1]{\left|#1\right|}
\newcommand{\lt}{<}
\newcommand{\gt}{>}
\newcommand{\amp}{&}
\begin{document}
\pagestyle{empty}
\resizebox{\width}{\height}{
 \tikzset{%
 }
\begin{tikzpicture} [xscale=.4]
 \draw[cyan] (0,0) grid (11,7);
 \draw[black,thick] (0,0)--++(11,0) node[below,, yshift=-.5cm,midway]{Clutch size};
 \draw[black,thick] (0,0)--(0,7);
 \node[rotate=90] at (-2.7,3.50) {Number of clutches};
 \foreach \x [evaluate=\x as \xi using int(1+\x)] in {1,3,...,11} {
  \node at ({\x-.5},-.25) {$\xi$};
 }
 \foreach \x [evaluate=\x as \xi using int(20*\x)] in {1,2,...,7} {
   \draw[black,thick] (.1,\x)--++(-.2,0) node[left, fill=white, inner sep=1, xshift=-2]{$\xi$};
 }
 \draw[red,fill=magenta!30!white] (0,0) rectangle ++(1,.05);
 \draw[red,fill=magenta!30!white] (2,0) rectangle ++(1,.1);
 \draw[red,fill=magenta!30!white] (3,0) rectangle ++(1,.6);
 \draw[red,fill=magenta!30!white] (4,0) rectangle ++(1,1.15);
 \draw[red,fill=magenta!30!white] (5,0) rectangle ++(1,3.6);
 \draw[red,fill=magenta!30!white] (6,0) rectangle ++(1,6.3);
 \draw[red,fill=magenta!30!white] (7,0) rectangle ++(1,5.75);
 \draw[red,fill=magenta!30!white] (8,0) rectangle ++(1,3);
 \draw[red,fill=magenta!30!white] (9,0) rectangle ++(1,1);
 \draw[red,fill=magenta!30!white] (10,0) rectangle ++(1,.15);
 \end{tikzpicture}
}
\end{document}
