\documentclass[10pt]{article}
\usepackage{geometry}
\geometry{letterpaper,total={340pt,9.0in}}
%% Custom Page Layout Adjustments (use publisher page-geometry)
\usepackage{amsmath,amssymb}
\usepackage{tikz}
\usetikzlibrary{calc}
\usetikzlibrary{arrows}
\usetikzlibrary{decorations.pathreplacing}
\usetikzlibrary{patterns}
\usetikzlibrary{calc,intersections,through,backgrounds}
\usetikzlibrary{shapes.geometric}
\ifdefined\tikzset
\tikzset{ampersand replacement = \amp}
\fi
\newcommand\degree[0]{^{\circ}}
\newcommand\Ccancel[2][black]{\renewcommand\CancelColor{\color{#1}}\cancel{#2}}
\newcommand{\alert}[1]{\boldsymbol{\color{magenta}{#1}}}
\newcommand{\blert}[1]{\boldsymbol{\color{blue}{#1}}}
\newcommand{\bluetext}[1]{\color{blue}{#1}} 
\delimitershortfall-1sp
\newcommand\abs[1]{\left|#1\right|}
\newcommand{\lt}{<}
\newcommand{\gt}{>}
\newcommand{\amp}{&}
\begin{document}
\pagestyle{empty}
\resizebox{\width}{\height}{
\tikzset{%
}
\begin{tikzpicture} [scale=.35]
\draw[cyan, thin] (-6,-7) grid (10,10);
\draw[black,very thick, ->, >=stealth'] (-6,0)--(10.8,0) node[right]{$x$};
\draw[black,very thick, ->, >=stealth'] (0,-7)--(0,10.8) node[above]{$y$};
\foreach \x in {-5,5,10} {
 \draw[black] (\x,.15) --++(0,-.3)  node[below, yshift=-2, fill=white, inner sep=1]   {$\x$};
  \draw[black] (.15,\x) --++(-.3,0)  node[left, xshift=-2, fill=white, inner sep=1]   {$\x$};
}
\filldraw[red] (-4,5) circle (.2) node[text=black, above left, yshift=2,fill=white, inner sep=2] {$A$};
\filldraw[red] (6,0) circle (.2) node[text=black, above right, yshift=2,fill=white, inner sep=2] {$B$};
\filldraw[red] (-2,-6) circle (.2) node[text=black, left, xshift=-2,fill=white, inner sep=2] {$C$};
\filldraw[red] (2,-3) circle (.2) node[text=black, below right, yshift=-2,fill=white, inner sep=2] {$P$};
\draw[blue, thick, dashed] (-4,5)--(2,-3);
\draw[blue] (2,-3)++({180+atan(-4/3)}:.7) -- ++({atan(3/4)}:.7) -- ++({atan(-4/3)}:.7);
\draw[magenta!90!black,ultra thick] (-4,5)--(6,0)--(-2,-6)--cycle;
\draw[blue, thick, dashed] (-4,5)--(2,-3);
\end{tikzpicture}
}
\end{document}
